\subsection*{Introduction}
This note is from the Number Theory class held at UC Berkeley in Fall 2023. The prerequisites for this course is Math 250 A, in particular, the following:
\begin{itemize}
    \item integrality of an element of a ring over a subring;
    \item integral ring extensions;
    \item separable and purely inseparable (algebraic) field extensions;
    \item Galois theory;
    \item noetherian rings and modules;
    \item localization (inverting a multiplicative subset of a ring).
\end{itemize}

\noindent
In this course, we will cover the following chapters of \textit{Algebraic Number Theory} of Neukirch.
\begin{itemize}
    \item Ch 1: Algebraic integers (all the sections but 12, 13, 14)
    \item Ch 2: Valuations (all the sections but 6, parts of 7, 9, 10)
    \item Ch 3: Primes, different, discriminant (1, 2, and parts of 3)
    \item Ch 7: Zeta functions and L-series (a thin subset)
    \item Ch 6: Class field theory (Section 12, a few other parts)
\end{itemize}

\subsection*{Overview}

The following is the overview of the courses. Let us define a number field.

\begin{definition}
    A number field is a finite (field) extension of $\Q$.
\end{definition}

\noindent
For example, $\Q(\sqrt{2})$ is a number field. We often work with one of the following situations:
\[\begin{tikzcd}
A \arrow[d, dash] \arrow[r, phantom, "\subseteq"] &K \arrow[d, dash] \\
\Z \arrow[r, phantom, "\subseteq"] &\Q
\end{tikzcd}
\quad\quad
\begin{tikzcd}
B \arrow[d, dash] \arrow[r, phantom, "\subseteq"] &L \arrow[d, dash] \\
A \arrow[r, phantom, "\subseteq"] &K
\end{tikzcd}
\]
\noindent
where $K, L$ are number fields. Here is an example to be proved later. If $K = \Q(\sqrt{2})$, in the left-hand diagram, then $A = \Z[\sqrt{2}]$.\\

In Chapter 1 Algebraic integers, we will consider a question. Which properties of $\Z$ remain true in $A$?
\begin{table}[h]
    \centering
    \begin{tabular}{|p{0.4\textwidth}|p{0.4\textwidth}|}
        \hline\\
        $\Z$ &  $A$\\
        \hline\\
        PID & Usually not PID but non-principality is determined by a finite group\\
        \hline
    \end{tabular}
    \label{tab:table1}
\end{table}
\clearpage