Monday, August 28, 2023.
\subsubsection*{Contents}
\begin{itemize}
    \item Reading for today: \S 1.2
    \item Integral ring extensions
\end{itemize}

\begin{proposition}
    \label{lecture3_1}
    Let $A\subseteq B$ be rings and let $b \in B$. Then TFAE:
    \begin{enumerate}
        \item[(i)] $b$ is integral over $A$;
        \item[(ii)] $A[b]$ is finite over $A$;
        \item[(iii)] There is a faithful module $M$ over $A[b]$ which is finitely generated as an $A$-module. A faithful module $M$ over $R$ is an $R$-module such that $\alpha M \neq 0$ for all nonzero $\alpha \in R$.
    \end{enumerate}
\end{proposition}
\begin{proof}
    For (i) $\Rightarrow$ (ii), assume that $b$ is integral over $A$. Let $f(b) = 0$ be an integral equation for $b$ over $A$ where $f(x)$ is a monic polynomial of degree $n$. Then $A[b]$ is generated as $A$-module by $1, b^2, \cdots, b^{n-1}$. For (ii) $\Rightarrow$ (iii), take $M = A[b]$.

    For (iii) $\Rightarrow$ (i), let $M$ be a faithful module over $A$ which is generated by $m_1, \cdots, m_n$. We can write 
    \[bm_i = c_{i1}m_1, \cdots, c_{in}m_n\]
    for all $i$ and define a matrix $C = (c_{ij})$ with the coefficients $c_{ij}$ in $A$. Let $f(x) = \det(xI_n - C)$ and $D = bI_n - C$. We realize that $f$ is a monic polynomial with coefficients in $A$.
    
    Let $\mathbf{m}$ be a column vector having $m_i$ as its $i$th row. Recall that $D^*D = DD^* = (\det D) I_n$ where $D^*$ is the adjoint matrix of $D$. Hence we have $D\mathbf{m} = b\mathbf{m} - b\mathbf{m} = 0$; hence $D^*D\mathbf{m} = 0$; hence $(\det D)\mathbf{m} = 0$; hence $(\det D)m_i = 0$ for all $m_i$. Since $M$ is faithful, we notice $f(b) = \det D = 0$. Therefore $b$ is integral over $A$.
\end{proof}

\clearpage
\begin{lemma}
    Let $A \subseteq B \subseteq C$ be rings. If $C$ is finite over $B$ and $B$ is finite over $A$, then $C$ is finite over $A$.
\end{lemma}
\begin{proof}
    \textbf{TO DO}
\end{proof}

\begin{lemma}
    Let $A \subseteq B$ be rings, and let $b_1, b_2 \in B$. If $b_1$ and $b_2$ are integral over $A$, then so is $b_1\pm b_2$ and $b_1b_2$.
\end{lemma}
\begin{proof}
    Since $b_2$ is integral over $A$, it's integral over $A[b_1]$. So $A[b_1, b_2]$ is finite over $A[b_1]$; so $A[b_1, b_2]$ is finite over $A$. Therefore $b_1 \pm b_2$ and $b_1b_2$ are integral over $A$ by Proposition \ref{lecture3_1}.
\end{proof}

\begin{corollary}
    Let $A \subseteq B$ be rings. Then the integral closure of $A$ in $B$ is a subring of $B$ and contains $A$.
\end{corollary}
\clearpage